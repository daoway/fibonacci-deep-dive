\documentclass[a4paper,12pt]{article}
\usepackage{amsmath,amssymb,amsfonts}
\usepackage{geometry}
\geometry{margin=1in}

\title{Fibonacci Numbers from the Maclaurin Series of the Generating Function}
\author{}
\date{}

\begin{document}
\maketitle

For a function \( f(x) \) that is infinitely differentiable at \( x=0 \),
its \textbf{Maclaurin series} is defined as
\[
f(x) = \sum_{n=0}^{\infty} \frac{f^{(n)}(0)}{n!} x^n.
\]
Hence, the coefficient of \(x^n\) in this expansion equals \( f^{(n)}(0)/n! \).

Let the Fibonacci sequence \(\{F_n\}\) satisfy
\[
F_0 = 0, \quad F_1 = 1, \quad F_{n} = F_{n-1} + F_{n-2}.
\]
Its generating function is
\[
F(x) = \sum_{n=0}^{\infty} F_n x^n.
\]
It can be shown that
\[
F(x) = \frac{x}{1 - x - x^2}.
\]

Expanding \(F(x)\) into a Maclaurin series gives
\[
F(x) = \sum_{n=0}^{\infty} \frac{F^{(n)}(0)}{n!} x^n.
\]
Comparing this with
\[
F(x) = \sum_{n=0}^{\infty} F_n x^n,
\]
we conclude that
\[
F_n = \frac{F^{(n)}(0)}{n!}.
\]

For \( n = 5 \),
\[
F_5 = \frac{F^{(5)}(0)}{5!}.
\]
Since the Fibonacci sequence begins as
\[
0, \; 1, \; 1, \; 2, \; 3, \; 5, \; 8, \dots,
\]
we have \(F_5 = 5\). Therefore,
\[
F^{(5)}(0) = 5! \times 5 = 600.
\]


\end{document}
