\documentclass[a4paper,12pt]{article}
\usepackage{amsmath,amssymb,amsfonts}
\usepackage{geometry}
\usepackage{listings}
\usepackage{xcolor}
\geometry{margin=1in}

\title{Fibonacci Numbers from the Maclaurin Series of the Generating Function}
\author{}
\date{}

\lstset{
	basicstyle=\ttfamily\footnotesize,
	backgroundcolor=\color{gray!5},
	frame=single,
	breaklines=true,
	postbreak=\mbox{\textcolor{red}{$\hookrightarrow$}\space},
	keywordstyle=\color{blue},
	commentstyle=\color{gray},
	showstringspaces=false
}

\begin{document}
	\maketitle
	
	For a function \( f(x) \) that is infinitely differentiable at \( x=0 \),
	its \textbf{Maclaurin series} is defined as
	\[
	f(x) = \sum_{n=0}^{\infty} \frac{f^{(n)}(0)}{n!} x^n.
	\]
	Hence, the coefficient of \(x^n\) in this expansion equals \( f^{(n)}(0)/n! \).
	
	Let the Fibonacci sequence \(\{F_n\}\) satisfy
	\[
	F_0 = 0, \quad F_1 = 1, \quad F_{n} = F_{n-1} + F_{n-2}.
	\]
	Its generating function is
	\[
	F(x) = \sum_{n=0}^{\infty} F_n x^n.
	\]
	As we have shown earlier:
	\[
	F(x) = \frac{x}{1 - x - x^2}.
	\]
	
	Expanding \(F(x)\) into a Maclaurin series gives
	\[
	F(x) = \sum_{n=0}^{\infty} \frac{F^{(n)}(0)}{n!} x^n.
	\]
	Comparing this with
	\[
	F(x) = \sum_{n=0}^{\infty} F_n x^n,
	\]
	we conclude that
	\[
	F_n = \frac{F^{(n)}(0)}{n!}.
	\]
	
	For \( n = 5 \),
	\[
	F_5 = \frac{F^{(5)}(0)}{5!}.
	\]
	
	We can perform a symbolic experiment to verify this relationship.
	
	\begin{lstlisting}[language=Python, caption={Symbolic verification in Python (SymPy)}]
		import sympy as sp
		
		# Define symbol and generating function
		x = sp.symbols('x')
		F = x / (1 - x - x**2)
		
		# Compute the 5th derivative
		F5_deriv = sp.diff(F, x, 5)
		
		# Evaluate at x = 0 and divide by 5!
		F5_at_0 = F5_deriv.subs(x, 0)
		F5_value = F5_at_0 / sp.factorial(5)
		
		print("F^(5)(x) =", sp.latex(F5_deriv))
		print("F^(5)(0)/5! =", F5_value)
		
		# Compare Fibonacci numbers from derivatives and standard definition
		def fib_from_derivative(n):
		Fn_deriv = sp.diff(F, x, n)
		Fn_at_0 = Fn_deriv.subs(x, 0)
		return int(Fn_at_0 / sp.factorial(n))
		
		def fib_iterative(n):
		if n <= 1:
		return n
		a, b = 0, 1
		for _ in range(2, n + 1):
		a, b = b, a + b
		return b
		
		for n in range(1, 11):
		print(n, fib_from_derivative(n), fib_iterative(n))
	\end{lstlisting}


This calculation has no practical purpose whatsoever.  
It is a purely theoretical (and slightly overcomplicated) way to verify something we already know perfectly well.  
It was done entirely for fun and curiosity, just to see that the Maclaurin series machinery indeed reproduces Fibonacci numbers symbolically.
	
\end{document}
