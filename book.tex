\documentclass{article}
\usepackage[margin=1in]{geometry}
\usepackage{amsmath,amsfonts}
\usepackage{listings}
\usepackage{fontspec}
\usepackage[bookmarks, hidelinks]{hyperref}
\usepackage{array} % For tables
\usepackage{graphicx} % 1. Include the graphicx package

% Set JetBrains Mono as the default monospaced font
\setmonofont{JetBrainsMono-Regular}

% Restore original IntelliJ-style code formatting
\lstdefinestyle{intelliJStyle}{
    language=Java,
    basicstyle=\ttfamily\small,
    numbers=left,
    numberstyle=\tiny,
    stepnumber=1,
    numbersep=5pt,
    frame=single,
    breaklines=true,
    breakatwhitespace=true,
    tabsize=1,
    showstringspaces=false,
    captionpos=b
}
\lstset{style=intelliJStyle}

\usepackage{listings}
% \usepackage{xcolor} % No longer strictly needed if only using black/white, but harmless to keep.

\lstdefinelanguage{JavaScript}{
	keywords={
		break, case, catch, class, const, continue, debugger, default, delete, do, 
		else, export, extends, finally, for, function, if, import, in, instanceof, 
		new, return, super, switch, this, throw, try, typeof, var, void, while, with, 
		yield, static, async, await, enum, implements, interface, let, package, private, 
		protected, public, arguments
	},
	morestring=[b]",
	morestring=[b]',
	morestring=[b]\`, % Template literal support
	comment=[l]//,
	comment=[s]{/*}{*/},
	ndkeywords={
		Array, Boolean, Date, Error, EvalError, Function, Infinity, JSON, Math, NaN, 
		Number, Object, Promise, Proxy, Reflect, RegExp, Set, String, Symbol, 
		TypeError, URIError, WeakMap, WeakSet, undefined, null, console, window, 
		document, alert, setTimeout, setInterval, clearTimeout, clearInterval, 
		fetch, require, module, process
	},
	keywordstyle=\bfseries, % Keywords will be bold
	ndkeywordstyle=\slshape, % Non-declared keywords will be slanted/italic
	stringstyle=\ttfamily, % Strings will be monospaced (typewriter) font
	commentstyle=\itshape\small, % Comments will be italic and slightly smaller
	identifierstyle=, % Default style for identifiers (usually regular text)
	sensitive=true
}

% Restore original Maple language definition
\lstdefinelanguage{Maple}{
    sensitive=true,
    morecomment=[l]{--},
    morecomment=[s]{/*}{*/},
    morestring=[b]",
    morestring=[b]',
    morekeywords={and,assuming,do,else,end,export,finally,for,if,implies,in,local,module,next,not,option,or,proc,quit,read,return,save,then,use,while},
    morekeywords=[2]{array,begin,by,case,description,elif,except,fi,proc,od,otherwise,repeat,return,select,then,until,when,where},
    morekeywords=[3]{diff,int,factor,integrate,limit,signum,sum},
    alsoletter={\$},
    literate=
        {>}{{\textgreater}}1
        {<}{{\textless}}1
}

\usepackage{titlesec}
\newcommand{\sectionbreak}{\clearpage}

\usepackage{float}

\begin{document}
\bibliographystyle{plain}

\title{Fibonacci Numbers: A Deep Dive}
\author{Stanislav Ostapenko}
\date{\today}
\maketitle

\begin{abstract}
	TODO
\end{abstract}

\clearpage

	\tableofcontents % Generate table of contents

	\clearpage
	
	\lstlistoflistings % Generate list of listings

\clearpage % or \newpage

\clearpage

	\thispagestyle{empty}

	\vspace*{\fill}
	\begin{center}
		\Huge
		\begin{align*}
			&   \mathcal{O}(1) &&= \mathcal{O}(\text{yeah})\\
			&    \mathcal{O}(\log_{} n) &&= \mathcal{O}(\text{nice})\\
			&    \mathcal{O}(n) &&= \mathcal{O}(\text{k})\\
			&    \mathcal{O}(n^{2}) &&= \mathcal{O}(\text{my})\\
			&    \mathcal{O}(2^{n}) &&= \mathcal{O}(\text{no})\\
			&    \mathcal{O}(n!) &&= \mathcal{O}(\text{mg})\\
			&    \mathcal{O}(n^{n}) &&= \mathcal{O}(\text{sh*t!})
		\end{align*}
	\end{center}
	\vspace*{\fill}

\clearpage

\section{Recursion and Mathematical Induction}
\section{Naive Recursion}
\subsection{Algorithm in Java}
\subsection{Binet's formula}
\subsubsection{Intuitive Explanation}
\subsubsection{Formal Derivation}
\subsection{Time Complexity (Big $\mathcal{O}$)}
\subsection{Empirical Validation of Time Complexity}
\subsection{Introduction to JVM Memory Structures}
\subsection{Method Execution in Java}
\subsubsection{Core concepts}
\subsubsection{Pass-by-Value in Java}
\subsubsection{Tail Recursion}
\subsection{Depth and \lstinline[basicstyle=\ttfamily\small]{StackOverflow}}
\subsection{Recursion Tree}
\subsection{Space Complexity}
\subsubsection{Call Stack Analysis}
\subsubsection{Clarification on Exponential Misconception}
\subsection{JVM Debugger view}
\subsection{Conclusion}
\section{Optimizing Recursion}
\subsection{Memoization}
\subsection{Dynamic Programming}
\subsection{Linear algebra and Fibonacci}
\subsubsection{Matrices and Transformations}
\subsubsection{Algorithm implementation in Java}
\subsubsection{Time and space complexity}
\subsection{Computer Algebra Systems}
\subsection{Binet formula in real life}
\subsection{Conclusion}
\section{Limitations of Primitive Types for Large Fibonacci Numbers}
\subsection{Showcase : from int to double}
\subsection{When double goes wild}
\subsubsection{$0.1 + 0.2 \ne 0.3$}
\subsubsection{IEEE 754 Representation}
\subsubsection{Calculate as machines}
\subsubsection{Conclusion}
\subsection{Handling Large Numbers with BigDecimal}
\section{Real-World Applications}
\subsection{Fibonacci heaps for Dijkstra's algorithm optimization}
\subsection{Fibonacci retracement levels in stock market analysis}
\subsection{Some of pseudorandom number generators}

%%
\end{document}
